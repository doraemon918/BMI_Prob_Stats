
% Default to the notebook output style

    


% Inherit from the specified cell style.




    
\documentclass[11pt]{article}

    
    
    \usepackage[T1]{fontenc}
    % Nicer default font (+ math font) than Computer Modern for most use cases
    \usepackage{mathpazo}

    % Basic figure setup, for now with no caption control since it's done
    % automatically by Pandoc (which extracts ![](path) syntax from Markdown).
    \usepackage{graphicx}
    % We will generate all images so they have a width \maxwidth. This means
    % that they will get their normal width if they fit onto the page, but
    % are scaled down if they would overflow the margins.
    \makeatletter
    \def\maxwidth{\ifdim\Gin@nat@width>\linewidth\linewidth
    \else\Gin@nat@width\fi}
    \makeatother
    \let\Oldincludegraphics\includegraphics
    % Set max figure width to be 80% of text width, for now hardcoded.
    \renewcommand{\includegraphics}[1]{\Oldincludegraphics[width=.8\maxwidth]{#1}}
    % Ensure that by default, figures have no caption (until we provide a
    % proper Figure object with a Caption API and a way to capture that
    % in the conversion process - todo).
    \usepackage{caption}
    \DeclareCaptionLabelFormat{nolabel}{}
    \captionsetup{labelformat=nolabel}

    \usepackage{adjustbox} % Used to constrain images to a maximum size 
    \usepackage{xcolor} % Allow colors to be defined
    \usepackage{enumerate} % Needed for markdown enumerations to work
    \usepackage{geometry} % Used to adjust the document margins
    \usepackage{amsmath} % Equations
    \usepackage{amssymb} % Equations
    \usepackage{textcomp} % defines textquotesingle
    % Hack from http://tex.stackexchange.com/a/47451/13684:
    \AtBeginDocument{%
        \def\PYZsq{\textquotesingle}% Upright quotes in Pygmentized code
    }
    \usepackage{upquote} % Upright quotes for verbatim code
    \usepackage{eurosym} % defines \euro
    \usepackage[mathletters]{ucs} % Extended unicode (utf-8) support
    \usepackage[utf8x]{inputenc} % Allow utf-8 characters in the tex document
    \usepackage{fancyvrb} % verbatim replacement that allows latex
    \usepackage{grffile} % extends the file name processing of package graphics 
                         % to support a larger range 
    % The hyperref package gives us a pdf with properly built
    % internal navigation ('pdf bookmarks' for the table of contents,
    % internal cross-reference links, web links for URLs, etc.)
    \usepackage{hyperref}
    \usepackage{longtable} % longtable support required by pandoc >1.10
    \usepackage{booktabs}  % table support for pandoc > 1.12.2
    \usepackage[inline]{enumitem} % IRkernel/repr support (it uses the enumerate* environment)
    \usepackage[normalem]{ulem} % ulem is needed to support strikethroughs (\sout)
                                % normalem makes italics be italics, not underlines
    

    
    
    % Colors for the hyperref package
    \definecolor{urlcolor}{rgb}{0,.145,.698}
    \definecolor{linkcolor}{rgb}{.71,0.21,0.01}
    \definecolor{citecolor}{rgb}{.12,.54,.11}

    % ANSI colors
    \definecolor{ansi-black}{HTML}{3E424D}
    \definecolor{ansi-black-intense}{HTML}{282C36}
    \definecolor{ansi-red}{HTML}{E75C58}
    \definecolor{ansi-red-intense}{HTML}{B22B31}
    \definecolor{ansi-green}{HTML}{00A250}
    \definecolor{ansi-green-intense}{HTML}{007427}
    \definecolor{ansi-yellow}{HTML}{DDB62B}
    \definecolor{ansi-yellow-intense}{HTML}{B27D12}
    \definecolor{ansi-blue}{HTML}{208FFB}
    \definecolor{ansi-blue-intense}{HTML}{0065CA}
    \definecolor{ansi-magenta}{HTML}{D160C4}
    \definecolor{ansi-magenta-intense}{HTML}{A03196}
    \definecolor{ansi-cyan}{HTML}{60C6C8}
    \definecolor{ansi-cyan-intense}{HTML}{258F8F}
    \definecolor{ansi-white}{HTML}{C5C1B4}
    \definecolor{ansi-white-intense}{HTML}{A1A6B2}

    % commands and environments needed by pandoc snippets
    % extracted from the output of `pandoc -s`
    \providecommand{\tightlist}{%
      \setlength{\itemsep}{0pt}\setlength{\parskip}{0pt}}
    \DefineVerbatimEnvironment{Highlighting}{Verbatim}{commandchars=\\\{\}}
    % Add ',fontsize=\small' for more characters per line
    \newenvironment{Shaded}{}{}
    \newcommand{\KeywordTok}[1]{\textcolor[rgb]{0.00,0.44,0.13}{\textbf{{#1}}}}
    \newcommand{\DataTypeTok}[1]{\textcolor[rgb]{0.56,0.13,0.00}{{#1}}}
    \newcommand{\DecValTok}[1]{\textcolor[rgb]{0.25,0.63,0.44}{{#1}}}
    \newcommand{\BaseNTok}[1]{\textcolor[rgb]{0.25,0.63,0.44}{{#1}}}
    \newcommand{\FloatTok}[1]{\textcolor[rgb]{0.25,0.63,0.44}{{#1}}}
    \newcommand{\CharTok}[1]{\textcolor[rgb]{0.25,0.44,0.63}{{#1}}}
    \newcommand{\StringTok}[1]{\textcolor[rgb]{0.25,0.44,0.63}{{#1}}}
    \newcommand{\CommentTok}[1]{\textcolor[rgb]{0.38,0.63,0.69}{\textit{{#1}}}}
    \newcommand{\OtherTok}[1]{\textcolor[rgb]{0.00,0.44,0.13}{{#1}}}
    \newcommand{\AlertTok}[1]{\textcolor[rgb]{1.00,0.00,0.00}{\textbf{{#1}}}}
    \newcommand{\FunctionTok}[1]{\textcolor[rgb]{0.02,0.16,0.49}{{#1}}}
    \newcommand{\RegionMarkerTok}[1]{{#1}}
    \newcommand{\ErrorTok}[1]{\textcolor[rgb]{1.00,0.00,0.00}{\textbf{{#1}}}}
    \newcommand{\NormalTok}[1]{{#1}}
    
    % Additional commands for more recent versions of Pandoc
    \newcommand{\ConstantTok}[1]{\textcolor[rgb]{0.53,0.00,0.00}{{#1}}}
    \newcommand{\SpecialCharTok}[1]{\textcolor[rgb]{0.25,0.44,0.63}{{#1}}}
    \newcommand{\VerbatimStringTok}[1]{\textcolor[rgb]{0.25,0.44,0.63}{{#1}}}
    \newcommand{\SpecialStringTok}[1]{\textcolor[rgb]{0.73,0.40,0.53}{{#1}}}
    \newcommand{\ImportTok}[1]{{#1}}
    \newcommand{\DocumentationTok}[1]{\textcolor[rgb]{0.73,0.13,0.13}{\textit{{#1}}}}
    \newcommand{\AnnotationTok}[1]{\textcolor[rgb]{0.38,0.63,0.69}{\textbf{\textit{{#1}}}}}
    \newcommand{\CommentVarTok}[1]{\textcolor[rgb]{0.38,0.63,0.69}{\textbf{\textit{{#1}}}}}
    \newcommand{\VariableTok}[1]{\textcolor[rgb]{0.10,0.09,0.49}{{#1}}}
    \newcommand{\ControlFlowTok}[1]{\textcolor[rgb]{0.00,0.44,0.13}{\textbf{{#1}}}}
    \newcommand{\OperatorTok}[1]{\textcolor[rgb]{0.40,0.40,0.40}{{#1}}}
    \newcommand{\BuiltInTok}[1]{{#1}}
    \newcommand{\ExtensionTok}[1]{{#1}}
    \newcommand{\PreprocessorTok}[1]{\textcolor[rgb]{0.74,0.48,0.00}{{#1}}}
    \newcommand{\AttributeTok}[1]{\textcolor[rgb]{0.49,0.56,0.16}{{#1}}}
    \newcommand{\InformationTok}[1]{\textcolor[rgb]{0.38,0.63,0.69}{\textbf{\textit{{#1}}}}}
    \newcommand{\WarningTok}[1]{\textcolor[rgb]{0.38,0.63,0.69}{\textbf{\textit{{#1}}}}}
    
    
    % Define a nice break command that doesn't care if a line doesn't already
    % exist.
    \def\br{\hspace*{\fill} \\* }
    % Math Jax compatability definitions
    \def\gt{>}
    \def\lt{<}
    % Document parameters
    \title{Homework 3}
    
    
    

    % Pygments definitions
    
\makeatletter
\def\PY@reset{\let\PY@it=\relax \let\PY@bf=\relax%
    \let\PY@ul=\relax \let\PY@tc=\relax%
    \let\PY@bc=\relax \let\PY@ff=\relax}
\def\PY@tok#1{\csname PY@tok@#1\endcsname}
\def\PY@toks#1+{\ifx\relax#1\empty\else%
    \PY@tok{#1}\expandafter\PY@toks\fi}
\def\PY@do#1{\PY@bc{\PY@tc{\PY@ul{%
    \PY@it{\PY@bf{\PY@ff{#1}}}}}}}
\def\PY#1#2{\PY@reset\PY@toks#1+\relax+\PY@do{#2}}

\expandafter\def\csname PY@tok@w\endcsname{\def\PY@tc##1{\textcolor[rgb]{0.73,0.73,0.73}{##1}}}
\expandafter\def\csname PY@tok@c\endcsname{\let\PY@it=\textit\def\PY@tc##1{\textcolor[rgb]{0.25,0.50,0.50}{##1}}}
\expandafter\def\csname PY@tok@cp\endcsname{\def\PY@tc##1{\textcolor[rgb]{0.74,0.48,0.00}{##1}}}
\expandafter\def\csname PY@tok@k\endcsname{\let\PY@bf=\textbf\def\PY@tc##1{\textcolor[rgb]{0.00,0.50,0.00}{##1}}}
\expandafter\def\csname PY@tok@kp\endcsname{\def\PY@tc##1{\textcolor[rgb]{0.00,0.50,0.00}{##1}}}
\expandafter\def\csname PY@tok@kt\endcsname{\def\PY@tc##1{\textcolor[rgb]{0.69,0.00,0.25}{##1}}}
\expandafter\def\csname PY@tok@o\endcsname{\def\PY@tc##1{\textcolor[rgb]{0.40,0.40,0.40}{##1}}}
\expandafter\def\csname PY@tok@ow\endcsname{\let\PY@bf=\textbf\def\PY@tc##1{\textcolor[rgb]{0.67,0.13,1.00}{##1}}}
\expandafter\def\csname PY@tok@nb\endcsname{\def\PY@tc##1{\textcolor[rgb]{0.00,0.50,0.00}{##1}}}
\expandafter\def\csname PY@tok@nf\endcsname{\def\PY@tc##1{\textcolor[rgb]{0.00,0.00,1.00}{##1}}}
\expandafter\def\csname PY@tok@nc\endcsname{\let\PY@bf=\textbf\def\PY@tc##1{\textcolor[rgb]{0.00,0.00,1.00}{##1}}}
\expandafter\def\csname PY@tok@nn\endcsname{\let\PY@bf=\textbf\def\PY@tc##1{\textcolor[rgb]{0.00,0.00,1.00}{##1}}}
\expandafter\def\csname PY@tok@ne\endcsname{\let\PY@bf=\textbf\def\PY@tc##1{\textcolor[rgb]{0.82,0.25,0.23}{##1}}}
\expandafter\def\csname PY@tok@nv\endcsname{\def\PY@tc##1{\textcolor[rgb]{0.10,0.09,0.49}{##1}}}
\expandafter\def\csname PY@tok@no\endcsname{\def\PY@tc##1{\textcolor[rgb]{0.53,0.00,0.00}{##1}}}
\expandafter\def\csname PY@tok@nl\endcsname{\def\PY@tc##1{\textcolor[rgb]{0.63,0.63,0.00}{##1}}}
\expandafter\def\csname PY@tok@ni\endcsname{\let\PY@bf=\textbf\def\PY@tc##1{\textcolor[rgb]{0.60,0.60,0.60}{##1}}}
\expandafter\def\csname PY@tok@na\endcsname{\def\PY@tc##1{\textcolor[rgb]{0.49,0.56,0.16}{##1}}}
\expandafter\def\csname PY@tok@nt\endcsname{\let\PY@bf=\textbf\def\PY@tc##1{\textcolor[rgb]{0.00,0.50,0.00}{##1}}}
\expandafter\def\csname PY@tok@nd\endcsname{\def\PY@tc##1{\textcolor[rgb]{0.67,0.13,1.00}{##1}}}
\expandafter\def\csname PY@tok@s\endcsname{\def\PY@tc##1{\textcolor[rgb]{0.73,0.13,0.13}{##1}}}
\expandafter\def\csname PY@tok@sd\endcsname{\let\PY@it=\textit\def\PY@tc##1{\textcolor[rgb]{0.73,0.13,0.13}{##1}}}
\expandafter\def\csname PY@tok@si\endcsname{\let\PY@bf=\textbf\def\PY@tc##1{\textcolor[rgb]{0.73,0.40,0.53}{##1}}}
\expandafter\def\csname PY@tok@se\endcsname{\let\PY@bf=\textbf\def\PY@tc##1{\textcolor[rgb]{0.73,0.40,0.13}{##1}}}
\expandafter\def\csname PY@tok@sr\endcsname{\def\PY@tc##1{\textcolor[rgb]{0.73,0.40,0.53}{##1}}}
\expandafter\def\csname PY@tok@ss\endcsname{\def\PY@tc##1{\textcolor[rgb]{0.10,0.09,0.49}{##1}}}
\expandafter\def\csname PY@tok@sx\endcsname{\def\PY@tc##1{\textcolor[rgb]{0.00,0.50,0.00}{##1}}}
\expandafter\def\csname PY@tok@m\endcsname{\def\PY@tc##1{\textcolor[rgb]{0.40,0.40,0.40}{##1}}}
\expandafter\def\csname PY@tok@gh\endcsname{\let\PY@bf=\textbf\def\PY@tc##1{\textcolor[rgb]{0.00,0.00,0.50}{##1}}}
\expandafter\def\csname PY@tok@gu\endcsname{\let\PY@bf=\textbf\def\PY@tc##1{\textcolor[rgb]{0.50,0.00,0.50}{##1}}}
\expandafter\def\csname PY@tok@gd\endcsname{\def\PY@tc##1{\textcolor[rgb]{0.63,0.00,0.00}{##1}}}
\expandafter\def\csname PY@tok@gi\endcsname{\def\PY@tc##1{\textcolor[rgb]{0.00,0.63,0.00}{##1}}}
\expandafter\def\csname PY@tok@gr\endcsname{\def\PY@tc##1{\textcolor[rgb]{1.00,0.00,0.00}{##1}}}
\expandafter\def\csname PY@tok@ge\endcsname{\let\PY@it=\textit}
\expandafter\def\csname PY@tok@gs\endcsname{\let\PY@bf=\textbf}
\expandafter\def\csname PY@tok@gp\endcsname{\let\PY@bf=\textbf\def\PY@tc##1{\textcolor[rgb]{0.00,0.00,0.50}{##1}}}
\expandafter\def\csname PY@tok@go\endcsname{\def\PY@tc##1{\textcolor[rgb]{0.53,0.53,0.53}{##1}}}
\expandafter\def\csname PY@tok@gt\endcsname{\def\PY@tc##1{\textcolor[rgb]{0.00,0.27,0.87}{##1}}}
\expandafter\def\csname PY@tok@err\endcsname{\def\PY@bc##1{\setlength{\fboxsep}{0pt}\fcolorbox[rgb]{1.00,0.00,0.00}{1,1,1}{\strut ##1}}}
\expandafter\def\csname PY@tok@kc\endcsname{\let\PY@bf=\textbf\def\PY@tc##1{\textcolor[rgb]{0.00,0.50,0.00}{##1}}}
\expandafter\def\csname PY@tok@kd\endcsname{\let\PY@bf=\textbf\def\PY@tc##1{\textcolor[rgb]{0.00,0.50,0.00}{##1}}}
\expandafter\def\csname PY@tok@kn\endcsname{\let\PY@bf=\textbf\def\PY@tc##1{\textcolor[rgb]{0.00,0.50,0.00}{##1}}}
\expandafter\def\csname PY@tok@kr\endcsname{\let\PY@bf=\textbf\def\PY@tc##1{\textcolor[rgb]{0.00,0.50,0.00}{##1}}}
\expandafter\def\csname PY@tok@bp\endcsname{\def\PY@tc##1{\textcolor[rgb]{0.00,0.50,0.00}{##1}}}
\expandafter\def\csname PY@tok@fm\endcsname{\def\PY@tc##1{\textcolor[rgb]{0.00,0.00,1.00}{##1}}}
\expandafter\def\csname PY@tok@vc\endcsname{\def\PY@tc##1{\textcolor[rgb]{0.10,0.09,0.49}{##1}}}
\expandafter\def\csname PY@tok@vg\endcsname{\def\PY@tc##1{\textcolor[rgb]{0.10,0.09,0.49}{##1}}}
\expandafter\def\csname PY@tok@vi\endcsname{\def\PY@tc##1{\textcolor[rgb]{0.10,0.09,0.49}{##1}}}
\expandafter\def\csname PY@tok@vm\endcsname{\def\PY@tc##1{\textcolor[rgb]{0.10,0.09,0.49}{##1}}}
\expandafter\def\csname PY@tok@sa\endcsname{\def\PY@tc##1{\textcolor[rgb]{0.73,0.13,0.13}{##1}}}
\expandafter\def\csname PY@tok@sb\endcsname{\def\PY@tc##1{\textcolor[rgb]{0.73,0.13,0.13}{##1}}}
\expandafter\def\csname PY@tok@sc\endcsname{\def\PY@tc##1{\textcolor[rgb]{0.73,0.13,0.13}{##1}}}
\expandafter\def\csname PY@tok@dl\endcsname{\def\PY@tc##1{\textcolor[rgb]{0.73,0.13,0.13}{##1}}}
\expandafter\def\csname PY@tok@s2\endcsname{\def\PY@tc##1{\textcolor[rgb]{0.73,0.13,0.13}{##1}}}
\expandafter\def\csname PY@tok@sh\endcsname{\def\PY@tc##1{\textcolor[rgb]{0.73,0.13,0.13}{##1}}}
\expandafter\def\csname PY@tok@s1\endcsname{\def\PY@tc##1{\textcolor[rgb]{0.73,0.13,0.13}{##1}}}
\expandafter\def\csname PY@tok@mb\endcsname{\def\PY@tc##1{\textcolor[rgb]{0.40,0.40,0.40}{##1}}}
\expandafter\def\csname PY@tok@mf\endcsname{\def\PY@tc##1{\textcolor[rgb]{0.40,0.40,0.40}{##1}}}
\expandafter\def\csname PY@tok@mh\endcsname{\def\PY@tc##1{\textcolor[rgb]{0.40,0.40,0.40}{##1}}}
\expandafter\def\csname PY@tok@mi\endcsname{\def\PY@tc##1{\textcolor[rgb]{0.40,0.40,0.40}{##1}}}
\expandafter\def\csname PY@tok@il\endcsname{\def\PY@tc##1{\textcolor[rgb]{0.40,0.40,0.40}{##1}}}
\expandafter\def\csname PY@tok@mo\endcsname{\def\PY@tc##1{\textcolor[rgb]{0.40,0.40,0.40}{##1}}}
\expandafter\def\csname PY@tok@ch\endcsname{\let\PY@it=\textit\def\PY@tc##1{\textcolor[rgb]{0.25,0.50,0.50}{##1}}}
\expandafter\def\csname PY@tok@cm\endcsname{\let\PY@it=\textit\def\PY@tc##1{\textcolor[rgb]{0.25,0.50,0.50}{##1}}}
\expandafter\def\csname PY@tok@cpf\endcsname{\let\PY@it=\textit\def\PY@tc##1{\textcolor[rgb]{0.25,0.50,0.50}{##1}}}
\expandafter\def\csname PY@tok@c1\endcsname{\let\PY@it=\textit\def\PY@tc##1{\textcolor[rgb]{0.25,0.50,0.50}{##1}}}
\expandafter\def\csname PY@tok@cs\endcsname{\let\PY@it=\textit\def\PY@tc##1{\textcolor[rgb]{0.25,0.50,0.50}{##1}}}

\def\PYZbs{\char`\\}
\def\PYZus{\char`\_}
\def\PYZob{\char`\{}
\def\PYZcb{\char`\}}
\def\PYZca{\char`\^}
\def\PYZam{\char`\&}
\def\PYZlt{\char`\<}
\def\PYZgt{\char`\>}
\def\PYZsh{\char`\#}
\def\PYZpc{\char`\%}
\def\PYZdl{\char`\$}
\def\PYZhy{\char`\-}
\def\PYZsq{\char`\'}
\def\PYZdq{\char`\"}
\def\PYZti{\char`\~}
% for compatibility with earlier versions
\def\PYZat{@}
\def\PYZlb{[}
\def\PYZrb{]}
\makeatother


    % Exact colors from NB
    \definecolor{incolor}{rgb}{0.0, 0.0, 0.5}
    \definecolor{outcolor}{rgb}{0.545, 0.0, 0.0}



    
    % Prevent overflowing lines due to hard-to-break entities
    \sloppy 
    % Setup hyperref package
    \hypersetup{
      breaklinks=true,  % so long urls are correctly broken across lines
      colorlinks=true,
      urlcolor=urlcolor,
      linkcolor=linkcolor,
      citecolor=citecolor,
      }
    % Slightly bigger margins than the latex defaults
    
    \geometry{verbose,tmargin=1in,bmargin=1in,lmargin=1in,rmargin=1in}
    
    

    \begin{document}
    
    
    \maketitle
    
    

    
    \hypertarget{homework-3}{%
\subsection{Homework 3}\label{homework-3}}

    \hypertarget{section}{%
\subparagraph{1.}\label{section}}

let's assume that two candidates are running in an election for Governor
of California. This fictitious election pits Mr.~Gubinator
vs.~Mr.~Ventura. We would like to know who is winning the race, and
therefore we conduct a poll of likely voters in California. If the poll
gives the voters a choice between the two candidates, then the results
can be reasonably modeled with the Binomial Distribution. States polls
indicate that 62\% of the population intend to vote for Mr.~Gubinator.

Calculate the binomial probabilities, and CF (confidence interval):

\begin{enumerate}
\def\labelenumi{\alph{enumi}.}
\tightlist
\item
  in a random sample of 50 we find 10 likely voters
\item
  in a random sample of 50 we find at least 3 likely voters
\item
  What is the average and standard deviation of the number of voters
  that we will find in our random sample?
\item
  plot the PMF and CMF from this random sample
\end{enumerate}

    \begin{Verbatim}[commandchars=\\\{\}]
{\color{incolor}In [{\color{incolor}2}]:} install.packages\PY{p}{(}\PY{l+s}{\PYZdq{}}\PY{l+s}{markovchain\PYZdq{}}\PY{p}{)}
        install.packages\PY{p}{(}\PY{l+s}{\PYZdq{}}\PY{l+s}{binom\PYZdq{}}\PY{p}{)}
        \PY{k+kn}{library}\PY{p}{(}binom\PY{p}{)}
\end{Verbatim}


    \begin{Verbatim}[commandchars=\\\{\}]
Updating HTML index of packages in '.Library'
Making 'packages.html' {\ldots} done
Updating HTML index of packages in '.Library'
Making 'packages.html' {\ldots} done

    \end{Verbatim}

    \begin{Verbatim}[commandchars=\\\{\}]
{\color{incolor}In [{\color{incolor}4}]:} \PY{l+s}{\PYZsq{}}\PY{l+s}{a\PYZsq{}}
        
        a\PY{o}{\PYZlt{}\PYZhy{}}dbinom\PY{p}{(}\PY{l+m}{10}\PY{p}{,}\PY{l+m}{50}\PY{p}{,}\PY{l+m}{0.62}\PY{p}{)}
        a
        
        \PY{l+s}{\PYZsq{}}\PY{l+s}{b\PYZsq{}}
        b\PY{o}{\PYZlt{}\PYZhy{}}\PY{p}{(}\PY{l+m}{1}\PY{o}{\PYZhy{}}pbinom\PY{p}{(}\PY{l+m}{3}\PY{p}{,}\PY{l+m}{50}\PY{p}{,}\PY{l+m}{0.62}\PY{p}{)}\PY{p}{)}
        b
\end{Verbatim}


    'a'

    
    1.33945315982445e-09

    
    'b'

    
    1

    
    \begin{Verbatim}[commandchars=\\\{\}]
{\color{incolor}In [{\color{incolor}35}]:} Average \PY{o}{\PYZlt{}\PYZhy{}} \PY{l+m}{50}\PY{o}{*}\PY{l+m}{0.62}
         Average
         sd \PY{o}{\PYZlt{}\PYZhy{}} \PY{k+kp}{sqrt}\PY{p}{(}Average\PY{o}{*}\PY{p}{(}\PY{l+m}{1}\PY{l+m}{\PYZhy{}0.62}\PY{p}{)}\PY{p}{)}
         sd
\end{Verbatim}


    31

    
    3.4322004603461

    
    \begin{Verbatim}[commandchars=\\\{\}]
{\color{incolor}In [{\color{incolor}36}]:} \PY{l+s}{\PYZsq{}}\PY{l+s}{c\PYZsq{}}
         binom.confint\PY{p}{(}\PY{l+m}{10}\PY{p}{,} \PY{l+m}{50}\PY{p}{,} conf.level \PY{o}{=} \PY{l+m}{0.95}\PY{p}{)}
         binom.confint\PY{p}{(}\PY{l+m}{3}\PY{p}{,} \PY{l+m}{50}\PY{p}{,} conf.level \PY{o}{=} \PY{l+m}{0.95}\PY{p}{)}
\end{Verbatim}


    'c'

    
    \begin{tabular}{r|llllll}
 method & x & n & mean & lower & upper\\
\hline
	 agresti-coull & 10            & 50            & 0.2000000     & 0.11050245    & 0.3323061    \\
	 asymptotic    & 10            & 50            & 0.2000000     & 0.08912769    & 0.3108723    \\
	 bayes         & 10            & 50            & 0.2058824     & 0.10116836    & 0.3171207    \\
	 cloglog       & 10            & 50            & 0.2000000     & 0.10318339    & 0.3196686    \\
	 exact         & 10            & 50            & 0.2000000     & 0.10030224    & 0.3371831    \\
	 logit         & 10            & 50            & 0.2000000     & 0.11113040    & 0.3332899    \\
	 probit        & 10            & 50            & 0.2000000     & 0.10792338    & 0.3279450    \\
	 profile       & 10            & 50            & 0.2000000     & 0.10570710    & 0.3242950    \\
	 lrt           & 10            & 50            & 0.2000000     & 0.10568422    & 0.3242911    \\
	 prop.test     & 10            & 50            & 0.2000000     & 0.10502158    & 0.3414368    \\
	 wilson        & 10            & 50            & 0.2000000     & 0.11243750    & 0.3303711    \\
\end{tabular}


    
    \begin{tabular}{r|llllll}
 method & x & n & mean & lower & upper\\
\hline
	 agresti-coull & 3             & 50            & 0.06000000    &  0.014420714  & 0.1683652    \\
	 asymptotic    & 3             & 50            & 0.06000000    & -0.005826784  & 0.1258268    \\
	 bayes         & 3             & 50            & 0.06862745    &  0.010699356  & 0.1376543    \\
	 cloglog       & 3             & 50            & 0.06000000    &  0.015682958  & 0.1488348    \\
	 exact         & 3             & 50            & 0.06000000    &  0.012548588  & 0.1654819    \\
	 logit         & 3             & 50            & 0.06000000    &  0.019480341  & 0.1701741    \\
	 probit        & 3             & 50            & 0.06000000    &  0.017542810  & 0.1581287    \\
	 profile       & 3             & 50            & 0.06000000    &  0.015303616  & 0.1482785    \\
	 lrt           & 3             & 50            & 0.06000000    &  0.015291827  & 0.1482746    \\
	 prop.test     & 3             & 50            & 0.06000000    &  0.015624594  & 0.1754187    \\
	 wilson        & 3             & 50            & 0.06000000    &  0.020614970  & 0.1621709    \\
\end{tabular}


    
    \begin{Verbatim}[commandchars=\\\{\}]
{\color{incolor}In [{\color{incolor}11}]:} \PY{l+s}{\PYZsq{}}\PY{l+s}{d\PYZsq{}}
         plot\PY{p}{(}dbinom\PY{p}{(}\PY{l+m}{0}\PY{o}{:}\PY{l+m}{50}\PY{p}{,}\PY{l+m}{50}\PY{p}{,}\PY{l+m}{0.62}\PY{p}{)}\PY{p}{,} type \PY{o}{=} \PY{l+s}{\PYZsq{}}\PY{l+s}{l\PYZsq{}}\PY{p}{)}
         plot\PY{p}{(}pbinom\PY{p}{(}\PY{l+m}{0}\PY{o}{:}\PY{l+m}{50}\PY{p}{,}\PY{l+m}{50}\PY{p}{,}\PY{l+m}{0.62}\PY{p}{)}\PY{p}{,} type \PY{o}{=} \PY{l+s}{\PYZsq{}}\PY{l+s}{s\PYZsq{}}\PY{p}{)}
\end{Verbatim}


    'd'

    
    \begin{center}
    \adjustimage{max size={0.9\linewidth}{0.9\paperheight}}{output_6_1.png}
    \end{center}
    { \hspace*{\fill} \\}
    
    \begin{center}
    \adjustimage{max size={0.9\linewidth}{0.9\paperheight}}{output_6_2.png}
    \end{center}
    { \hspace*{\fill} \\}
    
    \hypertarget{section}{%
\subparagraph{2.}\label{section}}

What are all the possible codon arrangements available during
translation in genetics? explain

    \begin{Verbatim}[commandchars=\\\{\}]
{\color{incolor}In [{\color{incolor}12}]:} \PY{l+s}{\PYZdq{}}\PY{l+s}{There are four proteins(ACTG) that will arrange in a set of three protein combinations such as ACT, GAC..., so the }
         \PY{l+s}{possible arrangements are calculated using permutation with repeat as ACTG\PYZca{}3:}
         
         \PY{l+s}{\PYZdq{}}
         \PY{l+m}{4}\PY{o}{\PYZca{}}\PY{l+m}{3}
\end{Verbatim}


    'There are four proteins(ACTG) that will arrange in a set of three protein combinations such as ACT, GAC..., so the 
possible arrangements are calculated using permutation with repeat as ACTG\textasciicircum{}3:

'

    
    64

    
    \hypertarget{section}{%
\subparagraph{3.}\label{section}}

Find the number of three-digit integers that have either 2 or 7 in the
tens digit and 2 in the units digit?

    \begin{Verbatim}[commandchars=\\\{\}]
{\color{incolor}In [{\color{incolor}13}]:} \PY{l+s}{\PYZsq{}}\PY{l+s}{3 digits: 9=(1\PYZhy{}9),2=(2 and 7), 1=(2)}
         \PY{l+s}{Calculation: 9*2*1:\PYZsq{}}
         \PY{l+m}{9}\PY{o}{*}\PY{l+m}{2}\PY{o}{*}\PY{l+m}{1}
\end{Verbatim}


    '3 digits: 9=(1-9),2=(2 and 7), 1=(2)
Calculation: 9*2*1:'

    
    18

    
    \hypertarget{section}{%
\subparagraph{4.}\label{section}}

Write a function in R that calculates nPk and nCk, and answer the
following questions keeping in mind the type of arrangement asked
(Permutations or combinations):

\begin{itemize}
\item
  \begin{enumerate}
  \def\labelenumi{\alph{enumi}.}
  \tightlist
  \item
    How many ways can we give awards to the first 3 places (gold,silver
    and bronze) among eight contestants?
  \end{enumerate}
\item
  \begin{enumerate}
  \def\labelenumi{\alph{enumi}.}
  \setcounter{enumi}{1}
  \tightlist
  \item
    Claire took 12 photos. She wants to put 5 of them on Facebook. How
    many groups of photos are possible?
  \end{enumerate}
\end{itemize}

    \begin{Verbatim}[commandchars=\\\{\}]
{\color{incolor}In [{\color{incolor}14}]:} \PY{l+s}{\PYZsq{}}\PY{l+s}{Permutation: order is revelant\PYZsq{}}
         nPk \PY{o}{\PYZlt{}\PYZhy{}} \PY{k+kr}{function}\PY{p}{(}n\PY{p}{,}k\PY{p}{)}\PY{p}{\PYZob{}}
             x\PY{o}{\PYZlt{}\PYZhy{}}\PY{k+kp}{factorial}\PY{p}{(}n\PY{p}{)}\PY{o}{/}\PY{k+kp}{factorial}\PY{p}{(}n\PY{o}{\PYZhy{}}k\PY{p}{)}
             \PY{k+kr}{return} \PY{p}{(}x\PY{p}{)}
         \PY{p}{\PYZcb{}}
         \PY{l+s}{\PYZsq{}}\PY{l+s}{Combinations: order is irrevelant\PYZsq{}}
         nCk \PY{o}{\PYZlt{}\PYZhy{}} \PY{k+kr}{function}\PY{p}{(}n\PY{p}{,}k\PY{p}{)}\PY{p}{\PYZob{}}
             x\PY{o}{\PYZlt{}\PYZhy{}}\PY{k+kp}{factorial}\PY{p}{(}n\PY{p}{)}\PY{o}{/}\PY{p}{(}\PY{k+kp}{factorial}\PY{p}{(}n\PY{o}{\PYZhy{}}k\PY{p}{)}\PY{o}{*}\PY{k+kp}{factorial}\PY{p}{(}k\PY{p}{)}\PY{p}{)}
             \PY{k+kr}{return}\PY{p}{(}x\PY{p}{)}
         \PY{p}{\PYZcb{}}
         
         \PY{l+s}{\PYZsq{}}\PY{l+s}{a is permutations\PYZsq{}}
         nPk\PY{p}{(}\PY{l+m}{8}\PY{p}{,}\PY{l+m}{3}\PY{p}{)}
         
         \PY{l+s}{\PYZsq{}}\PY{l+s}{b is combinations\PYZsq{}}
         nCk\PY{p}{(}\PY{l+m}{12}\PY{p}{,}\PY{l+m}{5}\PY{p}{)}
\end{Verbatim}


    'Permutation: order is revelant'

    
    'Combinations: order is irrevelant'

    
    'a is permutations'

    
    336

    
    'b is combinations'

    
    792

    
    \hypertarget{section}{%
\subparagraph{5.}\label{section}}

Seven friends went to watch a game. They were supposed to occupy seat
numbers 50 to 57. In how many different ways can they do it?

    \begin{Verbatim}[commandchars=\\\{\}]
{\color{incolor}In [{\color{incolor}16}]:} \PY{l+s}{\PYZdq{}}\PY{l+s}{since we can\PYZsq{}t have repeat on friend, it\PYZsq{}s permuatation where order is relevant\PYZdq{}}
         nPk\PY{p}{(}\PY{l+m}{7}\PY{p}{,}\PY{l+m}{7}\PY{p}{)}
\end{Verbatim}


    'since we can\textbackslash{}'t have repeat on friend, it's permuatation where order is relevant'

    
    5040

    
    \hypertarget{section}{%
\subparagraph{6.}\label{section}}

A famous example of a markov chain is the recruitment probabilities for
males in some ivy leave schools in the past, where Harvard, Dartmouth,
and Yale admitted only male students. Assume that, at that time, 80
percent of the sons of Harvard men went to Harvard and the rest went to
Yale, 40 percent of the sons of Yale men went to Yale, and the rest
split evenly between Harvard and Dartmouth; and of the sons of Dartmouth
men, 70 percent went to Dartmouth, 20 percent to Harvard, and 10 percent
to Yale.

Draw the transition matrix, plot the transition diagram and answer:

Calculate the probability that a grandson from Yale went to Harvard Draw
the transition matrix after 5 generations

    \begin{Verbatim}[commandchars=\\\{\}]
{\color{incolor}In [{\color{incolor}13}]:} \PY{k+kn}{library}\PY{p}{(}markovchain\PY{p}{)}
         school \PY{o}{\PYZlt{}\PYZhy{}} \PY{k+kt}{c}\PY{p}{(}\PY{l+s}{\PYZsq{}}\PY{l+s}{Havard\PYZsq{}}\PY{p}{,} \PY{l+s}{\PYZsq{}}\PY{l+s}{Dartmouth\PYZsq{}}\PY{p}{,} \PY{l+s}{\PYZsq{}}\PY{l+s}{Yale\PYZsq{}}\PY{p}{)}
         mtrix \PY{o}{\PYZlt{}\PYZhy{}} \PY{k+kt}{matrix}\PY{p}{(}data \PY{o}{=} \PY{k+kt}{c}\PY{p}{(}\PY{l+m}{0.8}\PY{p}{,} \PY{l+m}{0}\PY{p}{,} \PY{l+m}{0.2}\PY{p}{,}
                                 \PY{l+m}{0.2}\PY{p}{,} \PY{l+m}{0.7}\PY{p}{,} \PY{l+m}{0.1}\PY{p}{,}
                                 \PY{l+m}{0.3}\PY{p}{,} \PY{l+m}{0.3}\PY{p}{,} \PY{l+m}{0.4}\PY{p}{)}\PY{p}{,} byrow \PY{o}{=} \PY{n+nb+bp}{T}\PY{p}{,} nrow \PY{o}{=} \PY{l+m}{3}\PY{p}{,}
                                 dimnames \PY{o}{=} \PY{k+kt}{list}\PY{p}{(}school\PY{p}{,} school\PY{p}{)}\PY{p}{)}
         mcSchool \PY{o}{\PYZlt{}\PYZhy{}} new\PY{p}{(}\PY{l+s}{\PYZdq{}}\PY{l+s}{markovchain\PYZdq{}}\PY{p}{,} states \PY{o}{=} school\PY{p}{,} byrow \PY{o}{=} \PY{n+nb+bp}{T}\PY{p}{,}
                        transitionMatrix \PY{o}{=} mtrix\PY{p}{,} name \PY{o}{=} \PY{l+s}{\PYZdq{}}\PY{l+s}{Schools\PYZdq{}}\PY{p}{)}
         mtrix
         mcSchool
         \PY{c+c1}{\PYZsh{}plot(mcSchool)}
\end{Verbatim}


    \begin{tabular}{r|lll}
  & Havard & Dartmouth & Yale\\
\hline
	Havard & 0.8 & 0.0 & 0.2\\
	Dartmouth & 0.2 & 0.7 & 0.1\\
	Yale & 0.3 & 0.3 & 0.4\\
\end{tabular}


    
    
    \begin{verbatim}
Schools 
 A  3 - dimensional discrete Markov Chain defined by the following states: 
 Havard, Dartmouth, Yale 
 The transition matrix  (by rows)  is defined as follows: 
          Havard Dartmouth Yale
Havard       0.8       0.0  0.2
Dartmouth    0.2       0.7  0.1
Yale         0.3       0.3  0.4

    \end{verbatim}

    
    \begin{Verbatim}[commandchars=\\\{\}]
{\color{incolor}In [{\color{incolor}18}]:} \PY{l+s}{\PYZsq{}}\PY{l+s}{5th gen\PYZsq{}}
         fifthgen \PY{o}{\PYZlt{}\PYZhy{}} mcSchool\PY{o}{\PYZca{}}\PY{l+m}{5}
         fifthgen
         \PY{l+s}{\PYZdq{}}\PY{l+s}{probability of Yale 5th gen grandsons go to Harvard is 0.526, or 52.6\PYZpc{}\PYZdq{}}
\end{Verbatim}


    '5th gen'

    
    
    \begin{verbatim}
Schools^5 
 A  3 - dimensional discrete Markov Chain defined by the following states: 
 Havard, Dartmouth, Yale 
 The transition matrix  (by rows)  is defined as follows: 
           Havard Dartmouth    Yale
Havard    0.59006   0.17778 0.23216
Dartmouth 0.49883   0.29620 0.20497
Yale      0.52602   0.25935 0.21463

    \end{verbatim}

    
    'probability of Yale 5th gen grandsons go to Harvard is 0.526, or 52.6\%'

    

    % Add a bibliography block to the postdoc
    
    
    
    \end{document}
