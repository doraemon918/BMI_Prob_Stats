
% Default to the notebook output style

    


% Inherit from the specified cell style.




    
\documentclass[11pt]{article}

    
    
    \usepackage[T1]{fontenc}
    % Nicer default font (+ math font) than Computer Modern for most use cases
    \usepackage{mathpazo}

    % Basic figure setup, for now with no caption control since it's done
    % automatically by Pandoc (which extracts ![](path) syntax from Markdown).
    \usepackage{graphicx}
    % We will generate all images so they have a width \maxwidth. This means
    % that they will get their normal width if they fit onto the page, but
    % are scaled down if they would overflow the margins.
    \makeatletter
    \def\maxwidth{\ifdim\Gin@nat@width>\linewidth\linewidth
    \else\Gin@nat@width\fi}
    \makeatother
    \let\Oldincludegraphics\includegraphics
    % Set max figure width to be 80% of text width, for now hardcoded.
    \renewcommand{\includegraphics}[1]{\Oldincludegraphics[width=.8\maxwidth]{#1}}
    % Ensure that by default, figures have no caption (until we provide a
    % proper Figure object with a Caption API and a way to capture that
    % in the conversion process - todo).
    \usepackage{caption}
    \DeclareCaptionLabelFormat{nolabel}{}
    \captionsetup{labelformat=nolabel}

    \usepackage{adjustbox} % Used to constrain images to a maximum size 
    \usepackage{xcolor} % Allow colors to be defined
    \usepackage{enumerate} % Needed for markdown enumerations to work
    \usepackage{geometry} % Used to adjust the document margins
    \usepackage{amsmath} % Equations
    \usepackage{amssymb} % Equations
    \usepackage{textcomp} % defines textquotesingle
    % Hack from http://tex.stackexchange.com/a/47451/13684:
    \AtBeginDocument{%
        \def\PYZsq{\textquotesingle}% Upright quotes in Pygmentized code
    }
    \usepackage{upquote} % Upright quotes for verbatim code
    \usepackage{eurosym} % defines \euro
    \usepackage[mathletters]{ucs} % Extended unicode (utf-8) support
    \usepackage[utf8x]{inputenc} % Allow utf-8 characters in the tex document
    \usepackage{fancyvrb} % verbatim replacement that allows latex
    \usepackage{grffile} % extends the file name processing of package graphics 
                         % to support a larger range 
    % The hyperref package gives us a pdf with properly built
    % internal navigation ('pdf bookmarks' for the table of contents,
    % internal cross-reference links, web links for URLs, etc.)
    \usepackage{hyperref}
    \usepackage{longtable} % longtable support required by pandoc >1.10
    \usepackage{booktabs}  % table support for pandoc > 1.12.2
    \usepackage[inline]{enumitem} % IRkernel/repr support (it uses the enumerate* environment)
    \usepackage[normalem]{ulem} % ulem is needed to support strikethroughs (\sout)
                                % normalem makes italics be italics, not underlines
    

    
    
    % Colors for the hyperref package
    \definecolor{urlcolor}{rgb}{0,.145,.698}
    \definecolor{linkcolor}{rgb}{.71,0.21,0.01}
    \definecolor{citecolor}{rgb}{.12,.54,.11}

    % ANSI colors
    \definecolor{ansi-black}{HTML}{3E424D}
    \definecolor{ansi-black-intense}{HTML}{282C36}
    \definecolor{ansi-red}{HTML}{E75C58}
    \definecolor{ansi-red-intense}{HTML}{B22B31}
    \definecolor{ansi-green}{HTML}{00A250}
    \definecolor{ansi-green-intense}{HTML}{007427}
    \definecolor{ansi-yellow}{HTML}{DDB62B}
    \definecolor{ansi-yellow-intense}{HTML}{B27D12}
    \definecolor{ansi-blue}{HTML}{208FFB}
    \definecolor{ansi-blue-intense}{HTML}{0065CA}
    \definecolor{ansi-magenta}{HTML}{D160C4}
    \definecolor{ansi-magenta-intense}{HTML}{A03196}
    \definecolor{ansi-cyan}{HTML}{60C6C8}
    \definecolor{ansi-cyan-intense}{HTML}{258F8F}
    \definecolor{ansi-white}{HTML}{C5C1B4}
    \definecolor{ansi-white-intense}{HTML}{A1A6B2}

    % commands and environments needed by pandoc snippets
    % extracted from the output of `pandoc -s`
    \providecommand{\tightlist}{%
      \setlength{\itemsep}{0pt}\setlength{\parskip}{0pt}}
    \DefineVerbatimEnvironment{Highlighting}{Verbatim}{commandchars=\\\{\}}
    % Add ',fontsize=\small' for more characters per line
    \newenvironment{Shaded}{}{}
    \newcommand{\KeywordTok}[1]{\textcolor[rgb]{0.00,0.44,0.13}{\textbf{{#1}}}}
    \newcommand{\DataTypeTok}[1]{\textcolor[rgb]{0.56,0.13,0.00}{{#1}}}
    \newcommand{\DecValTok}[1]{\textcolor[rgb]{0.25,0.63,0.44}{{#1}}}
    \newcommand{\BaseNTok}[1]{\textcolor[rgb]{0.25,0.63,0.44}{{#1}}}
    \newcommand{\FloatTok}[1]{\textcolor[rgb]{0.25,0.63,0.44}{{#1}}}
    \newcommand{\CharTok}[1]{\textcolor[rgb]{0.25,0.44,0.63}{{#1}}}
    \newcommand{\StringTok}[1]{\textcolor[rgb]{0.25,0.44,0.63}{{#1}}}
    \newcommand{\CommentTok}[1]{\textcolor[rgb]{0.38,0.63,0.69}{\textit{{#1}}}}
    \newcommand{\OtherTok}[1]{\textcolor[rgb]{0.00,0.44,0.13}{{#1}}}
    \newcommand{\AlertTok}[1]{\textcolor[rgb]{1.00,0.00,0.00}{\textbf{{#1}}}}
    \newcommand{\FunctionTok}[1]{\textcolor[rgb]{0.02,0.16,0.49}{{#1}}}
    \newcommand{\RegionMarkerTok}[1]{{#1}}
    \newcommand{\ErrorTok}[1]{\textcolor[rgb]{1.00,0.00,0.00}{\textbf{{#1}}}}
    \newcommand{\NormalTok}[1]{{#1}}
    
    % Additional commands for more recent versions of Pandoc
    \newcommand{\ConstantTok}[1]{\textcolor[rgb]{0.53,0.00,0.00}{{#1}}}
    \newcommand{\SpecialCharTok}[1]{\textcolor[rgb]{0.25,0.44,0.63}{{#1}}}
    \newcommand{\VerbatimStringTok}[1]{\textcolor[rgb]{0.25,0.44,0.63}{{#1}}}
    \newcommand{\SpecialStringTok}[1]{\textcolor[rgb]{0.73,0.40,0.53}{{#1}}}
    \newcommand{\ImportTok}[1]{{#1}}
    \newcommand{\DocumentationTok}[1]{\textcolor[rgb]{0.73,0.13,0.13}{\textit{{#1}}}}
    \newcommand{\AnnotationTok}[1]{\textcolor[rgb]{0.38,0.63,0.69}{\textbf{\textit{{#1}}}}}
    \newcommand{\CommentVarTok}[1]{\textcolor[rgb]{0.38,0.63,0.69}{\textbf{\textit{{#1}}}}}
    \newcommand{\VariableTok}[1]{\textcolor[rgb]{0.10,0.09,0.49}{{#1}}}
    \newcommand{\ControlFlowTok}[1]{\textcolor[rgb]{0.00,0.44,0.13}{\textbf{{#1}}}}
    \newcommand{\OperatorTok}[1]{\textcolor[rgb]{0.40,0.40,0.40}{{#1}}}
    \newcommand{\BuiltInTok}[1]{{#1}}
    \newcommand{\ExtensionTok}[1]{{#1}}
    \newcommand{\PreprocessorTok}[1]{\textcolor[rgb]{0.74,0.48,0.00}{{#1}}}
    \newcommand{\AttributeTok}[1]{\textcolor[rgb]{0.49,0.56,0.16}{{#1}}}
    \newcommand{\InformationTok}[1]{\textcolor[rgb]{0.38,0.63,0.69}{\textbf{\textit{{#1}}}}}
    \newcommand{\WarningTok}[1]{\textcolor[rgb]{0.38,0.63,0.69}{\textbf{\textit{{#1}}}}}
    
    
    % Define a nice break command that doesn't care if a line doesn't already
    % exist.
    \def\br{\hspace*{\fill} \\* }
    % Math Jax compatability definitions
    \def\gt{>}
    \def\lt{<}
    % Document parameters
    \title{Homework 5}
    
    
    

    % Pygments definitions
    
\makeatletter
\def\PY@reset{\let\PY@it=\relax \let\PY@bf=\relax%
    \let\PY@ul=\relax \let\PY@tc=\relax%
    \let\PY@bc=\relax \let\PY@ff=\relax}
\def\PY@tok#1{\csname PY@tok@#1\endcsname}
\def\PY@toks#1+{\ifx\relax#1\empty\else%
    \PY@tok{#1}\expandafter\PY@toks\fi}
\def\PY@do#1{\PY@bc{\PY@tc{\PY@ul{%
    \PY@it{\PY@bf{\PY@ff{#1}}}}}}}
\def\PY#1#2{\PY@reset\PY@toks#1+\relax+\PY@do{#2}}

\expandafter\def\csname PY@tok@w\endcsname{\def\PY@tc##1{\textcolor[rgb]{0.73,0.73,0.73}{##1}}}
\expandafter\def\csname PY@tok@c\endcsname{\let\PY@it=\textit\def\PY@tc##1{\textcolor[rgb]{0.25,0.50,0.50}{##1}}}
\expandafter\def\csname PY@tok@cp\endcsname{\def\PY@tc##1{\textcolor[rgb]{0.74,0.48,0.00}{##1}}}
\expandafter\def\csname PY@tok@k\endcsname{\let\PY@bf=\textbf\def\PY@tc##1{\textcolor[rgb]{0.00,0.50,0.00}{##1}}}
\expandafter\def\csname PY@tok@kp\endcsname{\def\PY@tc##1{\textcolor[rgb]{0.00,0.50,0.00}{##1}}}
\expandafter\def\csname PY@tok@kt\endcsname{\def\PY@tc##1{\textcolor[rgb]{0.69,0.00,0.25}{##1}}}
\expandafter\def\csname PY@tok@o\endcsname{\def\PY@tc##1{\textcolor[rgb]{0.40,0.40,0.40}{##1}}}
\expandafter\def\csname PY@tok@ow\endcsname{\let\PY@bf=\textbf\def\PY@tc##1{\textcolor[rgb]{0.67,0.13,1.00}{##1}}}
\expandafter\def\csname PY@tok@nb\endcsname{\def\PY@tc##1{\textcolor[rgb]{0.00,0.50,0.00}{##1}}}
\expandafter\def\csname PY@tok@nf\endcsname{\def\PY@tc##1{\textcolor[rgb]{0.00,0.00,1.00}{##1}}}
\expandafter\def\csname PY@tok@nc\endcsname{\let\PY@bf=\textbf\def\PY@tc##1{\textcolor[rgb]{0.00,0.00,1.00}{##1}}}
\expandafter\def\csname PY@tok@nn\endcsname{\let\PY@bf=\textbf\def\PY@tc##1{\textcolor[rgb]{0.00,0.00,1.00}{##1}}}
\expandafter\def\csname PY@tok@ne\endcsname{\let\PY@bf=\textbf\def\PY@tc##1{\textcolor[rgb]{0.82,0.25,0.23}{##1}}}
\expandafter\def\csname PY@tok@nv\endcsname{\def\PY@tc##1{\textcolor[rgb]{0.10,0.09,0.49}{##1}}}
\expandafter\def\csname PY@tok@no\endcsname{\def\PY@tc##1{\textcolor[rgb]{0.53,0.00,0.00}{##1}}}
\expandafter\def\csname PY@tok@nl\endcsname{\def\PY@tc##1{\textcolor[rgb]{0.63,0.63,0.00}{##1}}}
\expandafter\def\csname PY@tok@ni\endcsname{\let\PY@bf=\textbf\def\PY@tc##1{\textcolor[rgb]{0.60,0.60,0.60}{##1}}}
\expandafter\def\csname PY@tok@na\endcsname{\def\PY@tc##1{\textcolor[rgb]{0.49,0.56,0.16}{##1}}}
\expandafter\def\csname PY@tok@nt\endcsname{\let\PY@bf=\textbf\def\PY@tc##1{\textcolor[rgb]{0.00,0.50,0.00}{##1}}}
\expandafter\def\csname PY@tok@nd\endcsname{\def\PY@tc##1{\textcolor[rgb]{0.67,0.13,1.00}{##1}}}
\expandafter\def\csname PY@tok@s\endcsname{\def\PY@tc##1{\textcolor[rgb]{0.73,0.13,0.13}{##1}}}
\expandafter\def\csname PY@tok@sd\endcsname{\let\PY@it=\textit\def\PY@tc##1{\textcolor[rgb]{0.73,0.13,0.13}{##1}}}
\expandafter\def\csname PY@tok@si\endcsname{\let\PY@bf=\textbf\def\PY@tc##1{\textcolor[rgb]{0.73,0.40,0.53}{##1}}}
\expandafter\def\csname PY@tok@se\endcsname{\let\PY@bf=\textbf\def\PY@tc##1{\textcolor[rgb]{0.73,0.40,0.13}{##1}}}
\expandafter\def\csname PY@tok@sr\endcsname{\def\PY@tc##1{\textcolor[rgb]{0.73,0.40,0.53}{##1}}}
\expandafter\def\csname PY@tok@ss\endcsname{\def\PY@tc##1{\textcolor[rgb]{0.10,0.09,0.49}{##1}}}
\expandafter\def\csname PY@tok@sx\endcsname{\def\PY@tc##1{\textcolor[rgb]{0.00,0.50,0.00}{##1}}}
\expandafter\def\csname PY@tok@m\endcsname{\def\PY@tc##1{\textcolor[rgb]{0.40,0.40,0.40}{##1}}}
\expandafter\def\csname PY@tok@gh\endcsname{\let\PY@bf=\textbf\def\PY@tc##1{\textcolor[rgb]{0.00,0.00,0.50}{##1}}}
\expandafter\def\csname PY@tok@gu\endcsname{\let\PY@bf=\textbf\def\PY@tc##1{\textcolor[rgb]{0.50,0.00,0.50}{##1}}}
\expandafter\def\csname PY@tok@gd\endcsname{\def\PY@tc##1{\textcolor[rgb]{0.63,0.00,0.00}{##1}}}
\expandafter\def\csname PY@tok@gi\endcsname{\def\PY@tc##1{\textcolor[rgb]{0.00,0.63,0.00}{##1}}}
\expandafter\def\csname PY@tok@gr\endcsname{\def\PY@tc##1{\textcolor[rgb]{1.00,0.00,0.00}{##1}}}
\expandafter\def\csname PY@tok@ge\endcsname{\let\PY@it=\textit}
\expandafter\def\csname PY@tok@gs\endcsname{\let\PY@bf=\textbf}
\expandafter\def\csname PY@tok@gp\endcsname{\let\PY@bf=\textbf\def\PY@tc##1{\textcolor[rgb]{0.00,0.00,0.50}{##1}}}
\expandafter\def\csname PY@tok@go\endcsname{\def\PY@tc##1{\textcolor[rgb]{0.53,0.53,0.53}{##1}}}
\expandafter\def\csname PY@tok@gt\endcsname{\def\PY@tc##1{\textcolor[rgb]{0.00,0.27,0.87}{##1}}}
\expandafter\def\csname PY@tok@err\endcsname{\def\PY@bc##1{\setlength{\fboxsep}{0pt}\fcolorbox[rgb]{1.00,0.00,0.00}{1,1,1}{\strut ##1}}}
\expandafter\def\csname PY@tok@kc\endcsname{\let\PY@bf=\textbf\def\PY@tc##1{\textcolor[rgb]{0.00,0.50,0.00}{##1}}}
\expandafter\def\csname PY@tok@kd\endcsname{\let\PY@bf=\textbf\def\PY@tc##1{\textcolor[rgb]{0.00,0.50,0.00}{##1}}}
\expandafter\def\csname PY@tok@kn\endcsname{\let\PY@bf=\textbf\def\PY@tc##1{\textcolor[rgb]{0.00,0.50,0.00}{##1}}}
\expandafter\def\csname PY@tok@kr\endcsname{\let\PY@bf=\textbf\def\PY@tc##1{\textcolor[rgb]{0.00,0.50,0.00}{##1}}}
\expandafter\def\csname PY@tok@bp\endcsname{\def\PY@tc##1{\textcolor[rgb]{0.00,0.50,0.00}{##1}}}
\expandafter\def\csname PY@tok@fm\endcsname{\def\PY@tc##1{\textcolor[rgb]{0.00,0.00,1.00}{##1}}}
\expandafter\def\csname PY@tok@vc\endcsname{\def\PY@tc##1{\textcolor[rgb]{0.10,0.09,0.49}{##1}}}
\expandafter\def\csname PY@tok@vg\endcsname{\def\PY@tc##1{\textcolor[rgb]{0.10,0.09,0.49}{##1}}}
\expandafter\def\csname PY@tok@vi\endcsname{\def\PY@tc##1{\textcolor[rgb]{0.10,0.09,0.49}{##1}}}
\expandafter\def\csname PY@tok@vm\endcsname{\def\PY@tc##1{\textcolor[rgb]{0.10,0.09,0.49}{##1}}}
\expandafter\def\csname PY@tok@sa\endcsname{\def\PY@tc##1{\textcolor[rgb]{0.73,0.13,0.13}{##1}}}
\expandafter\def\csname PY@tok@sb\endcsname{\def\PY@tc##1{\textcolor[rgb]{0.73,0.13,0.13}{##1}}}
\expandafter\def\csname PY@tok@sc\endcsname{\def\PY@tc##1{\textcolor[rgb]{0.73,0.13,0.13}{##1}}}
\expandafter\def\csname PY@tok@dl\endcsname{\def\PY@tc##1{\textcolor[rgb]{0.73,0.13,0.13}{##1}}}
\expandafter\def\csname PY@tok@s2\endcsname{\def\PY@tc##1{\textcolor[rgb]{0.73,0.13,0.13}{##1}}}
\expandafter\def\csname PY@tok@sh\endcsname{\def\PY@tc##1{\textcolor[rgb]{0.73,0.13,0.13}{##1}}}
\expandafter\def\csname PY@tok@s1\endcsname{\def\PY@tc##1{\textcolor[rgb]{0.73,0.13,0.13}{##1}}}
\expandafter\def\csname PY@tok@mb\endcsname{\def\PY@tc##1{\textcolor[rgb]{0.40,0.40,0.40}{##1}}}
\expandafter\def\csname PY@tok@mf\endcsname{\def\PY@tc##1{\textcolor[rgb]{0.40,0.40,0.40}{##1}}}
\expandafter\def\csname PY@tok@mh\endcsname{\def\PY@tc##1{\textcolor[rgb]{0.40,0.40,0.40}{##1}}}
\expandafter\def\csname PY@tok@mi\endcsname{\def\PY@tc##1{\textcolor[rgb]{0.40,0.40,0.40}{##1}}}
\expandafter\def\csname PY@tok@il\endcsname{\def\PY@tc##1{\textcolor[rgb]{0.40,0.40,0.40}{##1}}}
\expandafter\def\csname PY@tok@mo\endcsname{\def\PY@tc##1{\textcolor[rgb]{0.40,0.40,0.40}{##1}}}
\expandafter\def\csname PY@tok@ch\endcsname{\let\PY@it=\textit\def\PY@tc##1{\textcolor[rgb]{0.25,0.50,0.50}{##1}}}
\expandafter\def\csname PY@tok@cm\endcsname{\let\PY@it=\textit\def\PY@tc##1{\textcolor[rgb]{0.25,0.50,0.50}{##1}}}
\expandafter\def\csname PY@tok@cpf\endcsname{\let\PY@it=\textit\def\PY@tc##1{\textcolor[rgb]{0.25,0.50,0.50}{##1}}}
\expandafter\def\csname PY@tok@c1\endcsname{\let\PY@it=\textit\def\PY@tc##1{\textcolor[rgb]{0.25,0.50,0.50}{##1}}}
\expandafter\def\csname PY@tok@cs\endcsname{\let\PY@it=\textit\def\PY@tc##1{\textcolor[rgb]{0.25,0.50,0.50}{##1}}}

\def\PYZbs{\char`\\}
\def\PYZus{\char`\_}
\def\PYZob{\char`\{}
\def\PYZcb{\char`\}}
\def\PYZca{\char`\^}
\def\PYZam{\char`\&}
\def\PYZlt{\char`\<}
\def\PYZgt{\char`\>}
\def\PYZsh{\char`\#}
\def\PYZpc{\char`\%}
\def\PYZdl{\char`\$}
\def\PYZhy{\char`\-}
\def\PYZsq{\char`\'}
\def\PYZdq{\char`\"}
\def\PYZti{\char`\~}
% for compatibility with earlier versions
\def\PYZat{@}
\def\PYZlb{[}
\def\PYZrb{]}
\makeatother


    % Exact colors from NB
    \definecolor{incolor}{rgb}{0.0, 0.0, 0.5}
    \definecolor{outcolor}{rgb}{0.545, 0.0, 0.0}



    
    % Prevent overflowing lines due to hard-to-break entities
    \sloppy 
    % Setup hyperref package
    \hypersetup{
      breaklinks=true,  % so long urls are correctly broken across lines
      colorlinks=true,
      urlcolor=urlcolor,
      linkcolor=linkcolor,
      citecolor=citecolor,
      }
    % Slightly bigger margins than the latex defaults
    
    \geometry{verbose,tmargin=1in,bmargin=1in,lmargin=1in,rmargin=1in}
    
    

    \begin{document}
    
    
    \maketitle
    
    

    
    \hypertarget{homework-5}{%
\section{Homework 5}\label{homework-5}}

    \hypertarget{a-bulb-company-company-produces-high-efficient-bulbs.-the-probability-a-bulb-to-be-defective-is-0.01.}{%
\subsubsection{A bulb company company produces high efficient bulbs. The
probability a bulb to be defective is
0.01.}\label{a-bulb-company-company-produces-high-efficient-bulbs.-the-probability-a-bulb-to-be-defective-is-0.01.}}

\begin{enumerate}
\def\labelenumi{\alph{enumi}.}
\tightlist
\item
  What is the probability that a sample of 300 bulbs will contain
  exactly 5 defective?
\item
  Draw the histogram of probabilities
\end{enumerate}

    \begin{Verbatim}[commandchars=\\\{\}]
{\color{incolor}In [{\color{incolor}29}]:} \PY{l+s}{\PYZdq{}}
         \PY{l+s}{mean is 0.01*300 = 3, so the probability of 5 out of 300 is:}
         \PY{l+s}{\PYZdq{}}
         dpois\PY{p}{(}\PY{l+m}{5}\PY{p}{,}\PY{l+m}{3}\PY{p}{)}
         plot\PY{p}{(}\PY{l+m}{0}\PY{o}{:}\PY{l+m}{5}\PY{p}{,} dpois\PY{p}{(}\PY{l+m}{0}\PY{o}{:}\PY{l+m}{5}\PY{p}{,} \PY{l+m}{3}\PY{p}{)}\PY{p}{,} type \PY{o}{=} \PY{l+s}{\PYZdq{}}\PY{l+s}{h\PYZdq{}}\PY{p}{,} ylim \PY{o}{=} \PY{k+kt}{c}\PY{p}{(}\PY{l+m}{0}\PY{p}{,} \PY{l+m}{0.25}\PY{p}{)}\PY{p}{,}
             xlab \PY{o}{=} \PY{l+s}{\PYZdq{}}\PY{l+s}{X\PYZdq{}}\PY{p}{,} main \PY{o}{=} \PY{l+s}{\PYZdq{}}\PY{l+s}{hist\PYZdq{}}\PY{p}{,} ylab \PY{o}{=} \PY{l+s}{\PYZdq{}}\PY{l+s}{P(X)\PYZdq{}}\PY{p}{,} lwd \PY{o}{=} \PY{l+m}{3}\PY{p}{,} 
             cex.lab \PY{o}{=} \PY{l+m}{1.5}\PY{p}{,} cex.axis \PY{o}{=} \PY{l+m}{2}\PY{p}{,}
                                   cex.main \PY{o}{=} \PY{l+m}{2}\PY{p}{)}
\end{Verbatim}


    '
mean is 0.01*300 = 3, so the probability of 5 out of 300 is:
'

    
    0.100818813444924

    
    \begin{center}
    \adjustimage{max size={0.9\linewidth}{0.9\paperheight}}{output_2_2.png}
    \end{center}
    { \hspace*{\fill} \\}
    
    \hypertarget{b.-x-is-with-mean-ux3bc-40-find-and-plot-the-density-distribution-with-three-different-standard-deviations-2612-please-draw-the-curves-on-the-same-figure-with-different-colors-and-a-legend-for}{%
\subsubsection{B. X is with mean μ = 40, Find and plot the density
distribution with three different standard deviations (2,6,12) (Please
draw the curves on the same figure with different colors and a legend)
for:}\label{b.-x-is-with-mean-ux3bc-40-find-and-plot-the-density-distribution-with-three-different-standard-deviations-2612-please-draw-the-curves-on-the-same-figure-with-different-colors-and-a-legend-for}}

\begin{enumerate}
\def\labelenumi{\alph{enumi}.}
\tightlist
\item
  P(x \textless{} 40)
\item
  P(x \textgreater{} 21)
\end{enumerate}

    \begin{Verbatim}[commandchars=\\\{\}]
{\color{incolor}In [{\color{incolor}22}]:} \PY{l+s}{\PYZsq{}}\PY{l+s}{a\PYZsq{}}
         pnorm\PY{p}{(}\PY{l+m}{40}\PY{p}{,}\PY{l+m}{40}\PY{p}{,}\PY{l+m}{2}\PY{p}{)}
         pnorm\PY{p}{(}\PY{l+m}{40}\PY{p}{,}\PY{l+m}{40}\PY{p}{,}\PY{l+m}{6}\PY{p}{)}
         pnorm\PY{p}{(}\PY{l+m}{40}\PY{p}{,}\PY{l+m}{40}\PY{p}{,}\PY{l+m}{12}\PY{p}{)}
         a \PY{o}{\PYZlt{}\PYZhy{}} rnorm\PY{p}{(}\PY{l+m}{40}\PY{p}{,}\PY{l+m}{40}\PY{p}{,}\PY{l+m}{2}\PY{p}{)}
         b \PY{o}{\PYZlt{}\PYZhy{}} rnorm\PY{p}{(}\PY{l+m}{40}\PY{p}{,}\PY{l+m}{40}\PY{p}{,}\PY{l+m}{6}\PY{p}{)}
         \PY{k+kt}{c} \PY{o}{\PYZlt{}\PYZhy{}} rnorm\PY{p}{(}\PY{l+m}{40}\PY{p}{,}\PY{l+m}{40}\PY{p}{,}\PY{l+m}{12}\PY{p}{)}
         da \PY{o}{\PYZlt{}\PYZhy{}} density\PY{p}{(}a\PY{p}{)}
         db \PY{o}{\PYZlt{}\PYZhy{}} density\PY{p}{(}b\PY{p}{)}
         dc \PY{o}{\PYZlt{}\PYZhy{}} density\PY{p}{(}\PY{k+kt}{c}\PY{p}{)}
         plot\PY{p}{(}da\PY{p}{,} xlim \PY{o}{=}\PY{k+kt}{c}\PY{p}{(}\PY{l+m}{\PYZhy{}5}\PY{p}{,}\PY{l+m}{80}\PY{p}{)}\PY{p}{)}
         lines\PY{p}{(}db\PY{p}{,} col \PY{o}{=} \PY{l+s}{\PYZdq{}}\PY{l+s}{green\PYZdq{}}\PY{p}{)}
         lines\PY{p}{(}dc\PY{p}{,} col \PY{o}{=} \PY{l+s}{\PYZdq{}}\PY{l+s}{red\PYZdq{}}\PY{p}{)}
         legend\PY{p}{(}\PY{l+s}{\PYZsq{}}\PY{l+s}{topleft\PYZsq{}}\PY{p}{,} \PY{k+kt}{c}\PY{p}{(}\PY{l+s}{\PYZdq{}}\PY{l+s}{2sd\PYZdq{}}\PY{p}{,}\PY{l+s}{\PYZdq{}}\PY{l+s}{6sd\PYZdq{}}\PY{p}{,}\PY{l+s}{\PYZdq{}}\PY{l+s}{12sd\PYZdq{}}\PY{p}{)}\PY{p}{,} lty \PY{o}{=} \PY{k+kt}{c}\PY{p}{(}\PY{l+m}{1}\PY{p}{,}\PY{l+m}{3}\PY{p}{)}\PY{p}{,} col \PY{o}{=}\PY{k+kt}{c}\PY{p}{(}\PY{l+s}{\PYZdq{}}\PY{l+s}{black\PYZdq{}}\PY{p}{,}\PY{l+s}{\PYZdq{}}\PY{l+s}{green\PYZdq{}}\PY{p}{,} \PY{l+s}{\PYZdq{}}\PY{l+s}{red\PYZdq{}}\PY{p}{)}\PY{p}{)}
\end{Verbatim}


    'a'

    
    0.5

    
    0.5

    
    0.5

    
    \begin{center}
    \adjustimage{max size={0.9\linewidth}{0.9\paperheight}}{output_4_4.png}
    \end{center}
    { \hspace*{\fill} \\}
    
    \begin{Verbatim}[commandchars=\\\{\}]
{\color{incolor}In [{\color{incolor}24}]:} \PY{l+s}{\PYZsq{}}\PY{l+s}{b\PYZsq{}}
         pnorm\PY{p}{(}\PY{l+m}{21}\PY{p}{,}\PY{l+m}{40}\PY{p}{,}\PY{l+m}{2}\PY{p}{,} lower.tail \PY{o}{=} \PY{n+nb+bp}{F}\PY{p}{)}
         pnorm\PY{p}{(}\PY{l+m}{21}\PY{p}{,}\PY{l+m}{40}\PY{p}{,}\PY{l+m}{6}\PY{p}{,} lower.tail \PY{o}{=} \PY{n+nb+bp}{F}\PY{p}{)}
         pnorm\PY{p}{(}\PY{l+m}{21}\PY{p}{,}\PY{l+m}{40}\PY{p}{,}\PY{l+m}{12}\PY{p}{,} lower.tail \PY{o}{=} \PY{n+nb+bp}{F}\PY{p}{)}
         
         a \PY{o}{\PYZlt{}\PYZhy{}} rnorm\PY{p}{(}\PY{l+m}{21}\PY{p}{,}\PY{l+m}{40}\PY{p}{,}\PY{l+m}{2}\PY{p}{)}
         b \PY{o}{\PYZlt{}\PYZhy{}} rnorm\PY{p}{(}\PY{l+m}{21}\PY{p}{,}\PY{l+m}{40}\PY{p}{,}\PY{l+m}{6}\PY{p}{)}
         \PY{k+kt}{c} \PY{o}{\PYZlt{}\PYZhy{}} rnorm\PY{p}{(}\PY{l+m}{21}\PY{p}{,}\PY{l+m}{40}\PY{p}{,}\PY{l+m}{12}\PY{p}{)}
         da \PY{o}{\PYZlt{}\PYZhy{}} density\PY{p}{(}a\PY{p}{)}
         db \PY{o}{\PYZlt{}\PYZhy{}} density\PY{p}{(}b\PY{p}{)}
         dc \PY{o}{\PYZlt{}\PYZhy{}} density\PY{p}{(}\PY{k+kt}{c}\PY{p}{)}
         plot\PY{p}{(}da\PY{p}{,} xlim \PY{o}{=}\PY{k+kt}{c}\PY{p}{(}\PY{l+m}{0}\PY{p}{,}\PY{l+m}{70}\PY{p}{)}\PY{p}{)}
         lines\PY{p}{(}db\PY{p}{,} col \PY{o}{=} \PY{l+s}{\PYZdq{}}\PY{l+s}{green\PYZdq{}}\PY{p}{)}
         lines\PY{p}{(}dc\PY{p}{,} col \PY{o}{=} \PY{l+s}{\PYZdq{}}\PY{l+s}{red\PYZdq{}}\PY{p}{)}
         legend\PY{p}{(}\PY{l+s}{\PYZsq{}}\PY{l+s}{topleft\PYZsq{}}\PY{p}{,} \PY{k+kt}{c}\PY{p}{(}\PY{l+s}{\PYZdq{}}\PY{l+s}{2sd\PYZdq{}}\PY{p}{,}\PY{l+s}{\PYZdq{}}\PY{l+s}{6sd\PYZdq{}}\PY{p}{,}\PY{l+s}{\PYZdq{}}\PY{l+s}{12sd\PYZdq{}}\PY{p}{)}\PY{p}{,} lty \PY{o}{=} \PY{k+kt}{c}\PY{p}{(}\PY{l+m}{1}\PY{p}{,}\PY{l+m}{3}\PY{p}{)}\PY{p}{,} col \PY{o}{=}\PY{k+kt}{c}\PY{p}{(}\PY{l+s}{\PYZdq{}}\PY{l+s}{black\PYZdq{}}\PY{p}{,}\PY{l+s}{\PYZdq{}}\PY{l+s}{green\PYZdq{}}\PY{p}{,} \PY{l+s}{\PYZdq{}}\PY{l+s}{red\PYZdq{}}\PY{p}{)}\PY{p}{)}
\end{Verbatim}


    'b'

    
    1

    
    0.99922901521553

    
    0.943327245390237

    
    \begin{center}
    \adjustimage{max size={0.9\linewidth}{0.9\paperheight}}{output_5_4.png}
    \end{center}
    { \hspace*{\fill} \\}
    
    \hypertarget{c.-the-file-plant_heights.csv-contains-the-heights-for-1000-plants-chosen-along-a-altitudinal-gradient-from-low-to-high-elevation-in-a-natural-park-the-aim-of-the-study-is-to-understand-canopy-density-across-a-gradient.}{%
\subsubsection{C. The file plant\_heights.csv contains the heights for
1000 plants chosen along a altitudinal gradient (from low to high
elevation) in a natural park, the aim of the study is to understand
canopy density across a
gradient.}\label{c.-the-file-plant_heights.csv-contains-the-heights-for-1000-plants-chosen-along-a-altitudinal-gradient-from-low-to-high-elevation-in-a-natural-park-the-aim-of-the-study-is-to-understand-canopy-density-across-a-gradient.}}

Apply some of the descriptive statistics we have seem so far
(histograms, density plots, qqplots) to explain the behavior of the
data, write a summary paragraph that gives an initial overview of what
you see in this data (are the plants randomly distributed across the
gradient? what kind of distribution does the data have and how can you
explain it in a biological sense.

hint(try to plot the data using the log transformation, which will make
the patterns more evident)

    \begin{Verbatim}[commandchars=\\\{\}]
{\color{incolor}In [{\color{incolor}26}]:} plant \PY{o}{=} read.table\PY{p}{(}file \PY{o}{=} \PY{l+s}{\PYZdq{}}\PY{l+s}{plant\PYZus{}heights.csv\PYZdq{}}\PY{p}{,} header \PY{o}{=} \PY{n+nb+bp}{T}\PY{p}{)}
         \PY{k+kp}{tail}\PY{p}{(}plant\PY{p}{)}
\end{Verbatim}


    \begin{tabular}{r|l}
  & x\\
\hline
	9995 & 1198.191\\
	9996 & 1382.728\\
	9997 & 1453.086\\
	9998 & 1461.190\\
	9999 & 1790.739\\
	10000 & 1859.614\\
\end{tabular}


    
    \begin{Verbatim}[commandchars=\\\{\}]
{\color{incolor}In [{\color{incolor}30}]:} \PY{l+s}{\PYZdq{}}
         \PY{l+s}{histogram from original data:}
         \PY{l+s}{\PYZdq{}}
         hist\PY{p}{(}plant\PY{o}{\PYZdl{}}x\PY{p}{,}xlab \PY{o}{=} \PY{l+s}{\PYZdq{}}\PY{l+s}{heights\PYZdq{}}\PY{p}{,} ylab \PY{o}{=} \PY{l+s}{\PYZdq{}}\PY{l+s}{altitude\PYZdq{}}\PY{p}{,} main \PY{o}{=} \PY{k+kp}{expression}\PY{p}{(}italic\PY{p}{(}\PY{l+s}{\PYZdq{}}\PY{l+s}{Plant heights\PYZdq{}}\PY{p}{)}\PY{p}{)}\PY{p}{)}
         
         \PY{l+s}{\PYZdq{}}
         \PY{l+s}{from the original data, we can see that the higher the altitude, the higher the heights of the plants. From this graph, we can }
         \PY{l+s}{see that the distribution is skewer distribution.}
         \PY{l+s}{\PYZdq{}}
\end{Verbatim}


    '
histogram from original data:
'

    
    '
from the original data, we can see that the higher the altitude, the higher the heights of the plants. From this graph, we can 
see that the distribution is skewer distribution.
'

    
    \begin{center}
    \adjustimage{max size={0.9\linewidth}{0.9\paperheight}}{output_8_2.png}
    \end{center}
    { \hspace*{\fill} \\}
    
    \begin{Verbatim}[commandchars=\\\{\}]
{\color{incolor}In [{\color{incolor}34}]:} \PY{l+s}{\PYZdq{}}
         \PY{l+s}{Because the using the data, the histogram is too skewer, we need to}
         \PY{l+s}{apply transformation, in order for having better view of the data distribution:}
         \PY{l+s}{\PYZdq{}}
         plant\PY{o}{\PYZdl{}}log \PY{o}{=} \PY{k+kp}{log}\PY{p}{(}plant\PY{o}{\PYZdl{}}x\PY{p}{)}
         
         plant\PY{o}{\PYZdl{}}log10 \PY{o}{=} \PY{k+kp}{log}\PY{p}{(}plant\PY{o}{\PYZdl{}}x\PY{p}{,}\PY{l+m}{10}\PY{p}{)}
         hist\PY{p}{(}plant\PY{o}{\PYZdl{}}\PY{k+kp}{log}\PY{p}{,} main \PY{o}{=} \PY{k+kp}{expression}\PY{p}{(}italic\PY{p}{(}\PY{l+s}{\PYZdq{}}\PY{l+s}{Plant heights log\PYZdq{}}\PY{p}{)}\PY{p}{)}\PY{p}{)}
         
         \PY{l+s}{\PYZdq{}}
         \PY{l+s}{by taking the log, we can spread out the distribution }
         \PY{l+s}{\PYZdq{}}
\end{Verbatim}


    '
Because the using the data, the histogram is too skewer, we need to
apply transformation, in order for having better view of the data distribution:
'

    
    '
by taking the log, we can spread out the distribution 
'

    
    \begin{center}
    \adjustimage{max size={0.9\linewidth}{0.9\paperheight}}{output_9_2.png}
    \end{center}
    { \hspace*{\fill} \\}
    
    \begin{Verbatim}[commandchars=\\\{\}]
{\color{incolor}In [{\color{incolor}36}]:} par\PY{p}{(}mfrow \PY{o}{=} \PY{k+kt}{c}\PY{p}{(}\PY{l+m}{1}\PY{p}{,}\PY{l+m}{2}\PY{p}{)}\PY{p}{)}
         dplant \PY{o}{=} density\PY{p}{(}plant\PY{o}{\PYZdl{}}\PY{k+kp}{log}\PY{p}{)}
         plot\PY{p}{(}dplant\PY{p}{)}
         
         qqnorm\PY{p}{(}plant\PY{o}{\PYZdl{}}\PY{k+kp}{log}\PY{p}{)}
         qqline\PY{p}{(}plant\PY{o}{\PYZdl{}}\PY{k+kp}{log}\PY{p}{)}
         
         \PY{l+s}{\PYZdq{}}
         \PY{l+s}{These graphs showed density distribution and QQ plot.}
         \PY{l+s}{\PYZdq{}}
\end{Verbatim}


    \begin{center}
    \adjustimage{max size={0.9\linewidth}{0.9\paperheight}}{output_10_0.png}
    \end{center}
    { \hspace*{\fill} \\}
    

    % Add a bibliography block to the postdoc
    
    
    
    \end{document}
